%\section{Introduction}



The observation of ultra-high-energy cosmic rays (UHECRs) and ultra-high-energy neutrinos via the spectra from the Pierre Auger Collaboration or the IceCube Collaboration, respectively, has opened up several questions regarding their origin. Since their discovery in the early 20th century, cosmic rays have intrigued scientists and with the discovery of a spectra of ultra high energy cosmic rays and neutrinos their origins are now in question. The realization that these particles most probably come from extragalactic sources has turned our eyes to the deep cosmos and the most energetic objects in the universe. Knowing the sources of such particles can help us probe the most extreme environments, giving us new insight into unknown areas of physics.
 Knowing the sources of such particles can help us probe the most extreme environments, giving us new insight into unknown areas of physics. Progress in this area is capped by the low flux rate of the highest energy particles, but as better and better detectors are being built and planned, the future for multimessenger astronomy looks bright.

One of the best candidates for the production of UHECRs or high-energy neutrinos are active galactic nuclei (AGN). These are active galaxy cores found in the center of different galaxies spread across the night sky. AGN are known to be the most energetic objects in the universe and have suitable environments for the acceleration of particles to the highest energies, which has been studied in several papers such as \cite{PhysRevD.90.023007} and \cite{PhysRevLett.126.191101}. Of the different types of active galactic nuclei, a new source received a revival in the past year: compact symmetric objects (CSOs). Previously thought to be the precursors of larger radio galaxies, CSOs have now been shown in papers such as \cite{kiehlmann2023compact} to be a new class of jetted AGN. In this thesis, we will investigate the potential of CSOs as sources of UHECRs and neutrinos.










