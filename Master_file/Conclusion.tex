The main results of the thesis is that CSOs, more specifically CSO hotspots which are found in the jet termination area of CSO show great potential in being UHECRs sources, but not neutrino sources. 

We investigated the new list of CSO object produced by \cite{kiehlmann2023compact} and came to similar conclusion that they have sample the higher end of the luminosity function for CSOs. 

We have calculated the magnetic field of a selection of CSOs  included in the new list presented by \cite{kiehlmann2023compact}. The selection covered overlapping entries between the list compiled in \cite{kiehlmann2023compact} and a list of radio observation done by \cite{nrao1996}.  The hotspot from the overlapping sources vary from $0.01-1$ Gauss using two methods where the synchrotron self absorption method gives the highest value of magnetic field for sources that were imaged close to their turnover frequency. Mangetic equipartion estimated a magnetic field strenght between $0.01-0.1$ Gauss for most sources. The radius of our emitting region was estimated to be between $1-10$ pc for the same subsample of CSOs which coincides with previous results. Using these estimates we place the CSO hotspots in the hillas diagram showing them to be potent accelerators being able to reach energies of $10^{20}$ eV.
The minimum jet power and X-ray inferred jet power was calculated for a new subsample between \cite{kiehlmann2023compact} and \cite{W_jtowicz_2020}.  The minimu jet power of the stronges CSOs in the sample were of the order of mangnitude $10^{45}$ erg/s wheras the X-ray inferred sample measured a highest jet power of $2\times 10^{44}$ erg/s. The reuslts were reasonable and in agreement with  similar calculation found in \cite{W_jtowicz_2020}. We also calculated the X-ray emissivity of our sources using the estimated density of CSOs calculated as $1.2 \times 10^{-5}$ Mpc$^{-3}$ and the X-ray luminosity. This showed that compared to other AGNs the CSOs show a high or equivalent X-ray emissivty making them good candidates. 

Using conservative parameters calculated previously we estimated the characteristic timescales of a hypothetical proton being accelerated in the hotspot and showed that the proton would be able to reach energies of $10^{20}$ eV with the size of the system being the limiting factor. The timescales also showed that significant energloss mechanism were not present in the hotspot, making neutrino production unlikely via the photo-pion production mechanism.

Mass loading was introduced as a possible deceleration mechanism to explain the lifetimes of the different CSO 2 subclasses in previous literature. We used the mass loading as a potential source of protons to experience acceleration and showed that if all mass loading was converted to protons the Cosmic ray luminosity would be higher than the jet power of the CSOs making it unreasonable. We showed that the introduction of cold protons with a small fraction of protons being accelerated away could reduce the CR luminosity to a reasonable level, but would increase the energy density of ions making it higher than the magnetic field energy density. One concluded that the unconstrained variables in the mass loading model made it difficult to draw any conclusions, but that the model showed potential in using hotspots as UHECR sources.