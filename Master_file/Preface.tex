\section*{Preface}
This thesis is the result of my work collaborating with the  Astrophysics and Particle physics group at the University of Science and Technology in Norway as a part of my master thesis. This work has been done over one semester but is a somewhat continuation of a previous project done the semester before it with some sections including passages from the previous project. The work has been done in collaboration with my supervisor, Professor Foteini Oikonomou, and has been a great learning experience. I would like to thank her for her guidance and help throughout this project. I would also like to thank my fellow students for their help and support.

The thesis is centered around high energy particle creation and production and the enigmatic origins they have. For the last decades the observation of higher and higher energy cosmic rays and neutrinos have puzzled the scientific community, and the search for their origin still remains a mystery. This thesis aims on building upon previous work in this area by investigating a newly renewed class of active galactic nuclei, and using conterporary data to estimate the potential of these objects to produce high energy particles. Knowing the origins of these particles will allow us to test the ongoing theories of particle acceleration and give us new tools into probing the very sources producing them. Acceleration on the largest scales is a fascinating subject which I have come to appreciate more and more during this project. This thesis I hope will give a small contribution to the ongoing research in this field, and I hope that the reader will find it as interesting as I have.
