%\section{Discussion}

The methods used in this thesis have been able to show that CSOs hotspots are viable sources of UHECRs and not neutrinoes. The results are not as robust as we would like due to the low level of data found on CSOsand the unconstrained parameters used in chapter \ref{sec:emissivity_sec}, but we argue that they are indicative of the potential of CSOs. Given the new criteria for CSOs and the semi-novel idea of them being transient sources the possibility of them contributing to the local UHECR 
emissivity has increased. 

From the initial distribution of the known CSOs seen in chapter \ref{sec:CSO} it became clear that we are observing the most luminous part of their luminosity function. We have not derived a relation between the radio luminosity of the lobes and their luminosity of CR, but with more data this could be possible. The increase of detection of this subclass of AGN can be beneficial for the field of comsic ray origin, and the study of the most energetic particles in the universe. 

Given that CSOs, or at least CSO 2.0 and 2.1, can accelerate ions to the highest energies, we still have a problem of detection. According to the modern view of particle propagation through the interstellar medium (ISM) and the galactic medium (GM), particles such as protons and heavier elements will get scattered and deflected by the underlying magnetic field filling the ISM and GM. Therefore, the travel time of a photon and UHECRs from the same source will differ significantly.
As explained in \cite{Murase_2008} the average delay time for UHECRs is given by

\begin{equation}
    \bar{t}_d \approx \frac{D \theta^2 d}{4c} \simeq 10^5 \, \text{yrs} \left( \frac{D^2_{100 \, \text{Mpc}} B^2_{\text{IG},-9} \lambda_{\text{Mpc}}} {E_{20}^2}\right)
\end{equation}

Where $D$ is the distance to the source in units of Mpc, $E_{20}$ is the energy of the UHECR in units of $10^{20}$ eV, $B_{\text{IG}}$ is the strength of the intergalactic magnetic field in units of $10^{-9}$ G, and $\lambda$ is the coherence length of the magnetic field in Mpc. With characteristic values also obtained in \cite{Murase_2008} of $D = 50$ Mpc, $B_{\text{IG}} = 10^{-9}$ G, and $\lambda = 1$ Mpc, one sees that the average delay time is of the order of the lifetime of a CSO. 

This poses a challenge when groups such as the auger group try to retrace measured UHECRs to a CSO. The challenge is that one will no longer find that CSO. CSOs, being transient events with a lifetime of only a maximum of $(10^4)$ years, will possibly have disappeared from the night sky when their emitted UHECRs arrive at Earth. Therefore, the detection of UHECRs from CSOs could rely on the non-detection of any obvious sources when retracing the UHECRs. In addition to this, one could hope that the CSO will have some relic signature in the form of a radio-emitting region that has not yet disappeared, but this is no guarantee. The point of this passage is to highlight the possible difficulty in detecting UHECRs from CSOs and that the co-detection of both UHECRs and subsequent photons from the same source could be a rare event.

The SED modeling seen in chapter \ref{sec:photon_fields} have similarties of different groups who have done the same thing. In \cite{bronzini2024investigating} and in \cite{Ostorero_2010} the biggest difference comes in the contribution of IC and SSC emission. We explain this difference due to the fact that they did entire jet, lobe and core modeling of the emitting galaxy core, while we only focused on the hotspot. Our IC and SSC is the reuslt of only the electrons in the hotspot. We have added a factor 2 to their IC and SSC emission to account for two hotspots, but this is a very rough estimate. In future work we would like to incorporate the entire jet, and begin using observed data to constrain our model. For the moment the spectral fields created served their purpose and from small testing it is clear that an increase in IC and SSC emission would not change the photopion interaction significantly. The potential gain of using observational data to minimize the parameter space is clear. It would allow us to calculate upper limits on electron parameters, which one could expect protons to follow as well. As we have seen as well, the amount of power supplied in X-ray can also be used to constrain the amount of power in the jet further constraining the parameter space. Lastly the accretion rate can be constrained which would allow for better modeling of the interaction between the jet and the accretion disk.

The parameter estimation done in chapter \ref{sec:CSO_UHECR} showed that the different methods of determining size and mangetic field strength can vary quite largly between each other. This is a problem that is rooted in our data collection and the lack of observational data collected makes it hard for us to see any errortrends. The results still give us good estimations of the parameters, and have not limited the continued work in the project, but a bigger dataset with more clear constraints would be benificial. The magnetic field value of $0.01-1$ G is in line with other observed values of GHz peaked galaxies as measured in \cite{cheng2023highfrequency} which measured between $0.01 -0.05$ G. These values are more on the lower side, and shows why we chose to use conservative values further in our analysis.  

The radius estimates of our hotspots fall in line with what others have measrued as well but for GHz peaked galaxies. \cite{Perucho_2002} estiamtes a realtion between hotspot size and linear size of the source, and the order of magnetidue for his hotspots are $2-11$pc which is in line with our estimates. The same relation between linear size and hotspots can not be seen, but there is as mentioned before a large scatter in the data. The scatter is due to us not including viewing angle in our estimate of the hotspot, due to our assumption that it is spherical. The work from \cite{Perucho_2002} also show the same type of scatter that we see, where the linear size relation estimates on average a factor $10$ larger than the hotspot size. This has proven not to be a problem for our work, since when calculation the jet power of our sources, which is the only place we use this linear size relation the desired size is no longer only the hotspot, but the entire lobe. For future work, as always more data would be appreciated, and possible time variability estimates of the hotspots. 


The jet power which is affected by the estimation of the size and magnetic field values gave large values for the jet but they are within other measruements as made by \cite{W_jtowicz_2020} and are possible as written in \cite{readhead2023compact}. Therefore we can be farily confident in our fidnings. The reason for the X-ray inferred jet power being substantially lower, might be due to bad paramter estimation of the prefactor. In \cite{readhead2023compact} they show that the jetpower is related to the accretion rate via a factor $k \in (0.2-2)$ instead of $\eta_j \in (0.001-0.1)$ as we have used and is found in \cite{W_jtowicz_2020}. Both values estiamte the same thing which is accretion efficiency, and the values in \cite{readhead2023compact} are theoretical based on the rotation of the SMBH. Therefore we have more confidence using infeered efficiency as done in \cite{W_jtowicz_2020} but that the discrepancy between the two jet powers could be due to underestimation of the accretion efficiency. The other side of the coin is that we might have overestimated the minimum jet power, due to applying a the magnetic field measruements to the entire lobe. The calculation we have done are self consistent, but allowing for different energy densities in different parts of the jet, which does seems reasonable could yield a better estimation for the minimum jet power. Future work in this area would hopefully include some better constraint on the jet power, where one could do as explaine in \cite{Wykes_2013} of using enthalphy measruements or more direct measruements of the accretion disk, circumventing the need for x-ray to accretion disk power relations. Our result agree given our large errorbars, but the large errobars are indicative of the need for more and better constrained data on the individual sources. One parameter that might be worth measuring would be the shock speed or expansion speed as it is called. This would allow us to test other parameters in the minimum jet model and reach a more robust conclusion. 

The reuslt of Jet power put into future work can yield interesting result. Kinetic models such as used in \cite{sullivan2024smallscale} assume that all jet power is being counterbalanced with the ram pressure by the ambient medium, and that radiation losses are negligible. This assumption usually lead to the need for high ambient density or stellar mass loading as we have explored. But as seen in our result for the emissivity, a substantial sink of energy could be the acceleration of protons, and estimating a proper energy budget for the jet and its losses could, and i preface could lead to a need for more energy loss than is observed via radiation.

From our timescale analysis it became apparent that there are no large energy losses during the protons acceleration. The paramter that determined the cutoff was the size of our emitting system, and given that our model assumed a value on the lower end of our range $R = 2$ pc, it leads to the maximum possible energy achieved by protons is of the order $10^20$ eV. For future work in this area, creating the relation between maximum energy and size of the system good be interesting when calculating the contribution of CSOs to the spectra of UHECRs. With this hypothetical relation one can superimpose it on an luminosity function of CSOs and be able to more accurately estimate the contribution of CSOs to the UHECR emissivity, and its resulting spectrum. 

The timescale estimate also allowed us to estimate the resulting neutrino luminosity as a function of the proton luminosity, which showed that neutrinos are not very likely being produced in the hotspots of CSOs. The timescale for pion production is simply too long and the fraction of escaped protons producing neutrinos is too low. There is still areas of intereste that can produce neutrinos. In \cite{neronov2023neutrino} they discuss neutrino emission from seyfert galaxies, and in \cite{Kalashev_2023} they discuss neutrino emission from the central parsec of Blazar cores showing that the entire pciture of CSOs as neutrino produces has barley been scratched. 


Our simple analysis on mass loading and the subsequent acceleration of ions loaded into the lobe showed that there is possibly an equilirbium point between the ion density and the reuslting CR luminosity. For our results it is clear that the luminosity of protons is too high, and by adjusting our parameters we showed that the luminosity could be lowered but at the cost of a high ion energy density. This problem is not necessarly as big as one has stated since the values in which we are operating with are extremely unconstrained. The intial mass of the lobe which lead to the required mass loading are not unreasonable but not obersvationally bounded either. The result show if nothing else that the balance between cold protons and CR protons might be delicate, and the mass loading of the jet could be a key factor of introducing protons into the shocked area. The method of introducing protons via mass loading is also discuess in \cite{Wykes_2013} where they use a mass entrainment model of both ambient density and stellar loading. The result they acheive requries the particles to be heated in order to fix an imbalance of pressures as a result of mass loading. We have not moved further into such a model, but future work could include a better estimates on the effect of the mass entrainment on the lobe. 














 






